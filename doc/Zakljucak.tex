\chapter{Zaključak i budući rad}
		 
		 {Naš projektni zadatak bio je razvoj web aplikacije za rezervaciju treninga uz dodatak personalizirane ponude treninga ovisno o odabranim ciljevima korisnika. Provedba projekta bila je podijeljena u dvije faze. }
		 
		 {Prva faza projekta odvijala se u prvom ciklusu semestra. Glavni fokus bio je na okupljanju tima, raspodjeli poslova te dokumentaciji projekta. Prva revizija dokumentacija temeljila se na opisu projektnog zadatka, obrascima uporabe, dijagramima obrazaca uporabe, sekvencijskim dijagramima i dijagramima razreda. Definiranje funkcionalnih zahtjeva uvelike je olakšalo daljnji rad na aplikaciji. Također, izrada vizualnih prikaza idejnih rješenja aplikacije pomogla je svim članovima tima kod implementacije. }
		 
		 {Druga faza projekta odvijala se u drugom ciklusu semestra te je glavni fokus bio na programskom dijelu aplikacije. U odnosu na prvu fazu, članovi tima puno su više radili samostalno te su ssastanci bili puno rjeđi. Osim realizacije same aplikacije, u drugoj fazi potrebno je bilo dokumentirati ostatak UML dijagrama i provesti ispitivanje programskog rješenja. }
		 
		 {Jedina funkcionalnost koju nismo implementirali u potpunosti je ta da treneri unose pravila koje se odnose na rezervaciju termina svih korisnika. Naime, odlučili smo kako nam je puno jednostavnije da su ta pravila već određena u aplikaciji, a ne da ih trener sam unosi.}
		 
		 {Sudjelovanje na ovom projektu svim članovima tima bilo je jako korisno iskustvo. Imali smo priliku raditi u timu te smo na taj način naučili surađivati s drugim članovima i zajednički dolaziti do optimalnih rješenja. Zaključili smo da je najvažnija dobra komunikacija među članovima. Komunikacija među članovima tima uglavnom se odvijala preko Whatsappa te su tako svi članovi tima bili informirani o napretku projekta. Smatramo da smo vrlo dobro odradili zadatak, iako postoji prostora za napredak. }
		
		\eject 